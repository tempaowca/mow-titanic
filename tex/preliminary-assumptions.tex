\documentclass{article}

\usepackage{polski}
\usepackage[utf8]{inputenc}
\usepackage{indentfirst}

\begin{document}

\title{Metody odkrywania wiedzy \\%
{\large Klasyfikacja -- założenia wstępne} }

\author{Jakub Cichanowicz \and Artur Sawicki}

\maketitle

\section{Interpretacja tematu projektu}
Nasz temat związany jest z tragedią Titanica z 1912 roku. Statek podczas swojego pierwszego rejsu zderzył się z górą lodową, co spowodowało jego zatonięcie oraz śmierć większości osób znajdujących się na nim.

Prawda jest taka, że o przeżyciu nie zdecydowało tylko szczęście, ale także inne parametry, takie jak: wiek, płeć czy przynależność do grupy społecznej.

Do dyspozycji mamy dane zawierające 891 obserwacji i na ich podstawie spróbujemy stwierdzić od jakich parametrów głównie zależała szansa przeżycia w tej katastrofie.

\section{Opis algorytmów}

\section{Plan badań}

\subsection{Cel poszczególnych eksperymentów}
\subsection{Charakterystyka zbioru danych}
\subsection{Parametry algorytmów}
\subsection{Miary jakości i procedury oceny modeli}

\section{Otwarte kwestie}

\end{document}
