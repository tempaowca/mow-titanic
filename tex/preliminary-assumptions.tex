\documentclass{article}

\usepackage{polski}
\usepackage[utf8]{inputenc}
\usepackage{indentfirst}

\begin{document}

\title{Metody odkrywania wiedzy \\%
{\large Klasyfikacja -- założenia wstępne} }

\author{Jakub Cichanowicz \and Artur Sawicki}

\maketitle

\section{Interpretacja tematu projektu}
Nasz temat związany jest z tragedią Titanica z 1912 roku. Statek podczas swojego pierwszego rejsu zderzył się z górą lodową, co spowodowało jego zatonięcie oraz śmierć większości osób znajdujących się na nim.

Prawda jest taka, że o przeżyciu nie zdecydowało tylko szczęście, ale także inne parametry, takie jak: wiek, płeć czy przynależność do grupy społecznej.

Do dyspozycji mamy dane zawierające 891 obserwacji i na ich podstawie spróbujemy stwierdzić od jakich parametrów głównie zależała szansa przeżycia w tej katastrofie.

\section{Opis algorytmów}

\section{Plan badań}

\subsection{Cel poszczególnych eksperymentów}
\subsection{Charakterystyka zbioru danych}

W zbiorze danych mamy 12 zmiennych:
\begin{itemize}
\item Identyfikator pasażera ({\itshape PassengerId}) -- nie bierzemy jej pod uwagę, zmienna porządkowa;
\item Przeżycie ({\itshape Survived}) -- oznacza przeżycie danego pasażera, 0 - nie, 1 - tak, zmienna objaśniana;
\item Klasa pasażerska ({\itshape Pclass}) -- 1, 2 lub 3, nie ma braków danych, zmienna objaśniająca;
\item Imię i nazwisko z tytułem ({\itshape Name}) -- wstępnie wydaje nam się, że nie będziemy jej używać, jednak być może dałoby się jakoś z tego wyłączyć grupy ludzi po tytułach;
\item Płeć ({\itshape Sex}) -- nie ma braków danych, zmienna objaśniająca;
\item Wiek ({\itshape Age}) -- 177 braków, niektóre algorytmy sobie poradzą, inny sposób to dolosować, zastosujemy najprawdopodobniej oba podejścia, zmienna objaśniająca;
\item Liczba rodzeństwa/małżonków na pokładzie ({\itshape SibSp}) -- nie ma braków danych, zmienna objaśniająca;
\item Liczba dzieci/rodziców na pokładzie ({\itshape Parch}) -- nie ma braków danych, zmienna objaśniająca;
\item Numer biletu ({\itshape Ticket}) -- wstępnie wydaje się nieistotne, jednak być może będzie się dało jakoś pogrupować, być może przedrostki w biletach oznaczają jakieś częsci statku i będzie się dało to wykorzystać;
\item Opłata ({\itshape Fare}) -- nie ma braków danych, zmienna objaśniająca;
\item Kajuta ({\itshape Cabin}) -- 687 braków, spróbujemy wykorzystać obecność danych, aby lepiej klasyfikować osoby posiadające tę daną;
\item Miejsce wejścia na pokład ({\itshape Embarked}) -- nie wydaje się, żeby miało to znaczenie, ale może okazać się inaczej, zwłaszcza, że mamy bardzo mało braków, więc spróbujemy wykorzystać.
\end{itemize}

\subsection{Parametry algorytmów}
\subsection{Miary jakości i procedury oceny modeli}

\section{Otwarte kwestie}

\end{document}
